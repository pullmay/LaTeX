\documentclass[dvipdfmx,uplatex]{jsarticle}
\usepackage{color,amsmath,amssymb,enumerate,graphicx}
\usepackage[top=30truemm,bottom=30truemm,left=25truemm,right=25truemm]{geometry}
\usepackage{/Users/admin/Documents/LaTeX/style/mysty}%適切に指定する
\usepackage{/Users/admin/Documents/LaTeX/style/mythmj}%適切に指定する
\usepackage{tcolorbox}
\tcbuselibrary{skins}%shadow
\newtcolorbox{defbox}{enhanced, colback=white, colframe=black, sharp corners, boxrule=.75pt, drop shadow=black!30!white}
\newtcolorbox{theorembox}{enhanced, colback=white, colframe=black, boxrule=.75pt, arc=2mm, drop shadow=black!30!white}
%
\begin{document}
% \maketitle
% title
% \renewcommand{\thefootnote}{\fnsymbol{footnote}}
\begin{flushright}
  \begin{tabular}{rl}
    % 氏名: & miruca \footnotemark[2]\\
    作成: & 2020年2月28日\\
    更新: & \today
  \end{tabular}
\end{flushright}
%
\begin{center}
  % \vspace*{12pt}
  {\LARGE Title Comes Here}
  % \vspace{12pt}
\end{center}
% % abstract
% \abstract{Here comes abstract}
% % 注釈の番号をリセット
% \renewcommand{\thefootnote}{\arabic{footnote}}
% main document
\section{第1節}
%
\begin{tcolorbox}[enhanced,
    colframe=black, 
    colback=white,%背景の色
    % title = Title comes here,
    boxrule=.75pt, 
    % sharp corners,%角がカクカク
    drop fuzzy shadow=black!30!white,
    fonttitle = \bfseries]
\begin{lemma}
hogehoge
\end{lemma}
\end{tcolorbox}
%
\vspace{\baselineskip}
\begin{defbox}
\begin{definition}
  hoge
\end{definition}
\end{defbox}
%
\begin{proof}
prooooof.  
\end{proof}
%
\vspace{\baselineskip}
\begin{theorembox}
\begin{theorem}
  hoge
\end{theorem}
\end{theorembox}
%
\end{document}
