\documentclass[dvipdfmx,11pt]{beamer}
%----------------------------------------------------------------------------------%
\usetheme{default}
%----------------------------------------------------------------------------------%
\usepackage{amsmath}
%----------------------------------------------------------------------------------%
%Beamer色設定
%Black
\definecolor{AlmostBlack}{RGB}{38,38,38}
%Blue
\definecolor{UniBlue}{RGB}{0,150,200}
\definecolor{Navy}{RGB}{0,0,128}
\definecolor{DarkBlue}{RGB}{0,0,100}
%Red
\definecolor{AlertOrange}{RGB}{255,76,0}
\definecolor{FireBrick}{RGB}{178,34,34}
\definecolor{OrengeRed}{RGB}{255,69,0}
\definecolor{AlmostBlack}{RGB}{38,38,38}
%Green
\definecolor{DarkGreen}{RGB}{0,100,0}
%----------------------------------------------------------------------------------%
%beamercolorの設定
\setbeamercolor{normal text}{fg=AlmostBlack}%本文の色
\setbeamercolor{alerted text}{fg=OrengeRed}%\alertの色
\setbeamercolor{structure}{fg=Navy}%structreの色
\setbeamercolor{block title}{fg=white,bg=Navy}%ブロックのタイトルカラー
\setbeamercolor{block title alerted}{fg=white,bg=FireBrick}%ブロックのタイトルカラー
\setbeamercolor{block title example}{fg=white,bg=DarkGreen}%ブロックのタイトルカラー
\setbeamercolor*{palette secondary}{use=structure,fg=white,bg=DarkBlue} %上部の色
%----------------------------------------------------------------------------------%
\setbeamertemplate{navigation symbols}{} %ナビゲーションをなくす
\setbeamertemplate{footline}[frame number]{}%ページ番号
% \setbeamercolor{page number in head/foot}{fg=black, bg=black}%ページ番号の色
\setbeamerfont{page number in head/foot}{family=\ttfamily}%ページ番号のフォント・大きさ
% \setbeamerfont{page number in head/foot}{family=\ttfamily,size=\scriptsize}%ページ番号のフォント・大きさ
\setbeamertemplate{page number in head/foot}[totalframenumber]
%----------------------------------------------------------------------------------%
%%% 和文用 %%%
\usepackage{bxdpx-beamer}% ナビゲーションシンボル
\usepackage{pxjahyper}
\usepackage{minijs}
%%% 和文用 ここまで %%%
%----------------------------------------------------------------------------------%
\usepackage{bookmark}
\usepackage{helvet}
\renewcommand{\kanjifamilydefault}{\gtdefault} % 既定をゴシック体に
\usepackage[deluxe,uplatex]{otf} % 日本語多ウェイト化
% \renewcommand\thefootnote{*\arabic{footnote})}%脚注のスタイル
% \renewcommand{\baselinestretch}{1.1}%行間を調整
%----------------------------------------------------------------------------------%
\usefonttheme{professionalfonts}% 数式
\setbeamerfont{title}{size=\LARGE} % タイトル文字サイズ
\setbeamerfont{author}{size=\normalsize} % タイトル文字サイズ
\setbeamerfont{institute}{size=\scriptsize} % 所属の文字サイズ
\setbeamerfont{date}{size=\scriptsize} % 日付の文字サイズ
\setbeamerfont{frametitle}{size=\large} % フレームタイトルの文字サイズ
\setbeamerfont{footnote}{size=\scriptsize} % 脚注の文字サイズを小さくする
\useinnertheme{circles} % 箇条書きをシンプルにする
% \setbeamertemplate{background}[grid][step=10mm]%背景を方眼に
%----------------------------------------------------------------------------------%
\AtBeginSection[]{%各セクションごとに目次を生成
  \begin{frame}[plain]\frametitle{}
    \setbeamertemplate{section in toc}[sections numbered]
    \tableofcontents[currentsection]
  \end{frame}
}
%----------------------------------------------------------------------------------%
% information
\title{双対性理論}
\date{\today}
\author{miruca}
\institute{%
Graduate School of Informatics, Kyoto University
}
%----------------------------------------------------------------------------------%
%----------------------------------------------------------------------------------%
%----------------------------------------------------------------------------------%
%----------------------------------------------------------------------------------%
%----------------------------------------------------------------------------------%
%----------------------------------------------------------------------------------%
%----------------------------------------------------------------------------------%
%----------------------------------------------------------------------------------%
%----------------------------------------------------------------------------------%
%----------------------------------------------------------------------------------%
%----------------------------------------------------------------------------------%
%----------------------------------------------------------------------------------%
%----------------------------------------------------------------------------------%
\begin{document}
%----------------------------------------------------------------------------------%
%----------------------------------------------------------------------------------%
%----------------------------------------------------------------------------------%
% \maketitle%表紙の生成
\begin{frame}[plain]\frametitle{}
\titlepage %表紙
\end{frame}
%----------------------------------------------------------------------------------%
\begin{frame}{Today's Topic}%目次の生成
  \setbeamertemplate{section in toc}[sections numbered]
  \tableofcontents[hideallsubsections]
\end{frame}
%----------------------------------------------------------------------------------%
%----------------------------------------------------------------------------------%
%----------------------------------------------------------------------------------%
\section{ミニマックス問題と鞍点}
%----------------------------------------------------------------------------------%
\begin{frame}[t]\frametitle{ミニマックス問題}%t->top,c->center,b->bottom
    \begin{itemize}
      \item $2$つの最適化問題の関係について調べる
    \end{itemize}
\end{frame}
%----------------------------------------------------------------------------------%
%----------------------------------------------------------------------------------%
%----------------------------------------------------------------------------------%
\section{Section2}
%----------------------------------------------------------------------------------%
\begin{frame}[t]\frametitle{ブロック環境}
    \begin{itemize}
      \item 3種類のブロック環境が用意されている.
    \end{itemize}
    \begin{block}{block}
      通常のブロック環境.
    \end{block}
    %
    \begin{alertblock}{alert block}
      定義などで用いられるイメージ.
    \end{alertblock}
    %
    \begin{exampleblock}{example block}
      例などで用いられるイメージ.
    \end{exampleblock}
\end{frame}
%----------------------------------------------------------------------------------%
%----------------------------------------------------------------------------------%
%----------------------------------------------------------------------------------%
\section{Section3}
%----------------------------------------------------------------------------------%
\begin{frame}[t]\frametitle{数式}
    \begin{itemize}
      \item ``\texttt{professionalfonts}'' を使用している
    \end{itemize}
    \begin{align}
        \int^{b}_{a} f(x) \,dx = \lim_{n \to \infty} \sum^{n-1}_{i=1} f(x_{i}) \Delta x
      \end{align}
\end{frame}
%----------------------------------------------------------------------------------%
%----------------------------------------------------------------------------------%
%----------------------------------------------------------------------------------%
\section{Section4}
%----------------------------------------------------------------------------------%
\begin{frame}[t,allowframebreaks]\frametitle{参考文献}
%斜体にすると警告が出るのでとりあえずは"\textit"を使っていない.
    \scriptsize{%
    \begin{thebibliography}{99}
    \setlength{\itemsep}{-.5zw}
    \beamertemplatetextbibitems
    %
    \bibitem{MultT} Kotz S, Nadarajah S. Multivariate t-distributions and their applications. Cambridge University Press, 2004.
    %
    \end{thebibliography}
    }
\end{frame}
%----------------------------------------------------------------------------------%
%----------------------------------------------------------------------------------%
%----------------------------------------------------------------------------------%
%----------------------------------------------------------------------------------%
%----------------------------------------------------------------------------------%
\end{document}
